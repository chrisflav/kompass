

\documentclass{article}

\usepackage[utf8]{inputenc}
\usepackage{booktabs}
\usepackage{amssymb}
\usepackage{cmbright}
\usepackage{graphicx}
\usepackage{textpos}
\usepackage[colorlinks, breaklinks]{hyperref}
\usepackage{float}
\usepackage[margin=1in]{geometry}
\usepackage{array}
\usepackage{tabularx}
\usepackage{ltablex}

\usepackage{tikz}
\usepackage{setspace}
\usepackage{pbox}

\newcommand{\picpos}[4]{
  \begin{textblock*}{#1}(#2, #3)
    \includegraphics[width=\textwidth]{#4}
  \end{textblock*}
}

% set formatting
\setlength\parindent{0pt}
\setlength\parskip{0.6em plus 0.1em minus 0.1em}

% custom url command for properly formatting emails
\DeclareUrlCommand\Email{\urlstyle{same}}
% allow linebreak after every character
\expandafter\def\expandafter\UrlBreaks\expandafter{\UrlBreaks  
\do\/\do\a\do\b\do\c\do\d\do\e\do\f\do\g\do\h\do\i\do\j\do\k
\do\l\do\m\do\n\do\o\do\p\do\q\do\r\do\s\do\t\do\u\do\v
\do\w\do\x\do\y\do\z
\do\A\do\B\do\C\do\D\do\E\do\F\do\G\do\H\do\I\do\J\do\K
\do\L\do\M\do\N\do\O\do\P\do\Q\do\R\do\S\do\T\do\U\do\V
\do\W\do\X\do\Y\do\Z}

\renewcommand{\arraystretch}{1}

\newcolumntype{L}{>{\hspace{0pt}}X}
\newcommand{\tickedbox}{
    \makebox[0pt][l]{$\square$}\raisebox{.15ex}{\hspace{0.1em}$\checkmark$}
}
\newcommand{\checkbox}{
    \makebox[0pt][l]{$\square$}
}

% algemeines Layout Formular-Elemente
\renewcommand*{\DefaultOptionsofRadio}{print,radio, radiosymbol=6, width=\baselineskip, bordercolor={black}, borderwidth=0pt}
\renewcommand*{\DefaultOptionsofText}{print,bordercolor={black}, backgroundcolor=white, borderwidth=0pt}

\renewcommand{\LayoutTextField}[2]{% label, field
\setbox0=\hbox{#1\unskip}\ifdim\wd0=0pt
\setbox1=\hbox{#2\unskip}\ifdim\ht1>3ex
% Multiline
\begin{tikzpicture}[every node/.style={inner sep=0,outer sep=0}]
  \node[anchor=west] (TextFieldNode) at (0cm,0cm) {#2};
  \draw [thick] (current bounding box.south west) rectangle (current bounding box.north east);
\end{tikzpicture}%
\else
% Inline field, lowered a little bit to be better integrated into the text
\raisebox{-3.2pt}{\begin{tikzpicture}[every node/.style={inner sep=0,outer sep=0}]
  \node[anchor=west] (TextFieldNode) at (0cm,0cm) {#2};
  \draw[thick] ([yshift=-0.3ex]TextFieldNode.south west) -- ([yshift=-0.3ex]TextFieldNode.south east);
\end{tikzpicture}}%
\fi
\else
% Field with label below it
\begin{tikzpicture}[every node/.style={inner sep=0,outer sep=0}]
    \node[anchor=west] (TextFieldNode) at (0cm,2ex) {#2};
    \draw[thick] ([yshift=-0.3ex]TextFieldNode.south west) -- ([yshift=-0.3ex]TextFieldNode.south east);
    \node[anchor=west,font=\small] at (0cm,-0.9ex) {#1};
\end{tikzpicture}%
\fi
}

\newcommand{\radiosize}{0.33cm}
\newcommand{\yesnoticks}[1]{%
\raisebox{-3.2pt}{\begin{tikzpicture}[every node/.style={inner sep=0,outer sep=0}]
  \node[anchor=west,style={inner sep=2px}] (FieldYes) at (0cm,0cm) {\ChoiceMenu[radio=true,name=#1,width=\radiosize,height=\radiosize]{}{=Yes}};
  \node[anchor=west] (LabelYes) at ([xshift=0.7ex]FieldYes.east) {Ja};
  \node[anchor=west,style={inner sep=2px}] (FieldNo) at ([xshift=1ex]LabelYes.east) {\ChoiceMenu[radio=true,name=#1,width=\radiosize,height=\radiosize]{}{=No}};
  \node[anchor=west] (LabelNo) at ([xshift=0.7ex]FieldNo.east) {Nein};
  \draw [thick] ([xshift=-(\radiosize+0.15cm),yshift=-(\radiosize+0.15cm)]FieldYes.north east) rectangle (FieldYes.north east);
  \draw [thick] ([xshift=-(\radiosize+0.15cm),yshift=-(\radiosize+0.15cm)]FieldNo.north east) rectangle (FieldNo.north east);
\end{tikzpicture}}%
}

% Formularfeld, leer, editierbar
\newcommand{\field}[2]{\TextField[width=#2]{#1}}

% Formularfeld, vorbefüllt, editierbar
\newcommand{\fieldpf}[3]{\TextField[width=#2,value=#3]{#1}}

% Formularfeld, vorbefüllt, nicht editierbar
\newcommand{\fieldpfro}[3]{\TextField[width=#2,value=#3,readonly]{#1}}

% Formularfeld, in den Fließtext integriert
\newcommand{\fieldinline}[2]{\TextField[width=#2,name=#1]{}}

% Ja-Nein Antwort
\newcommand{\yesno}[2]{\pbox{0.8\textwidth}{\setstretch{1}#1}\hfill\yesnoticks{#2}}

% Dummy-Formularfeld: Sieht genauso aus, ist aber statisch, vorbefüllt, nicht editierbar.
\newcommand{\fieldd}[3]{% Label, width, prefilled text
\begin{tikzpicture}[every node/.style={inner sep=0,outer sep=0}]
    % Draw the static field rectangle
    \node[anchor=west] (DummyFieldNode) at (0cm,2ex) {\strut #3}; % Prefilled text
    \draw[thick] ([xshift=0cm,yshift=-0.3ex]DummyFieldNode.south west) -- ([xshift=#2,yshift=-0.3ex]DummyFieldNode.south west);

    % Label underneath
    \node[anchor=west,font=\small] at (0cm,-1.2ex) {#1};
\end{tikzpicture}%
}


\begin{document}
% HEADER RIGHT
\picpos{4.5cm}{11.5cm}{0cm}{/app/jdav_web/static/general/img/dav_logo_hd.png}
\begin{textblock*}{5cm}(11.5cm, 2.3cm)
    \begin{flushright}
        \small
        \noindent Deutscher Alpenverein e. V. \\
        Sektion {{ settings.SEKTION }} \\
        {{ settings.SEKTION_STREET }} \\
        {{ settings.SEKTION_TOWN }} \\
        Tel.: {{ settings.SEKTION_TELEPHONE }} \\
        Fax: {{ settings.SEKTION_TELEFAX }} \\
        {{ settings.SEKTION_CONTACT_MAIL }} \\
    \end{flushright}
\end{textblock*}

% HEADLINE


\textbf{\LARGE Anmeldung und Einverständniserklärung}

\textbf{Wir melden unser Kind verbindlich zur Jugendgruppe an:}

\begin{Form}
    \fieldd{Vor- und Nachname des Kindes}{0.65\linewidth}{ {{ member.name|esc_all }} }

    \fieldd{Geburtsdatum}{0.3\linewidth}{ {{ member.birth_date_str|esc_all }} }  \hfill
    \fieldd{Gender}{0.3\linewidth}{ {{ member.gender_str|esc_all }} }   \hspace{0.35\linewidth}

    \fieldpf{DAV-Mitglieds-Nr.}{0.3\linewidth}{ {{ member.dav_badge_no|esc_all }} }   \hfill
    \fieldd{Telefon (mobil)*}{0.3\linewidth}{ {{ member.phone_number|esc_all }} }   \hspace{0.35\linewidth}

    \fieldd{Adresse}{0.65\linewidth}{ {{ member.street|esc_all }}, {{ member.plz|esc_all }} {{ member.town|esc_all }} }

    \fieldd{E-Mail}{0.65\linewidth}{ {{ member.email|esc_all }} }

{\small *wenn vorhanden, Kontaktdaten des Kindes, ansonsten einer Kontaktperson}

\textbf{Notfallkontakte:}

\begin{tabular*}{\linewidth}{@{\extracolsep{\fill}}lll}
    Vollständiger Name & E-Mail & Telefon (mobil) \\
    \midrule

{{ c.name|esc_all }} & {{ c.email|esc_all }} & {{ c.phone_number|esc_all }} \\

\bottomrule
\end{tabular*}

\textbf{Medizinische Informationen:}

Gemäß den Teilnahmebedingungen verpflichten wir uns, vollständige
Angaben zu etwaigen gesundheitlichen Einschränkungen oder Erkrankungen unseres
Kindes zu machen:

\field{Allergien}{\linewidth}

\field{Medikamenten-Einnahmen}{\linewidth}

\field{Weitere Informationen zu Einschränkungen}{\linewidth}

\yesno{Unser Kind hat eine aktuelle Tetanus-Schutzimpfung}{tetanus}

\yesno{Unser Kind hat sichere Schwimmkenntnisse}{schwimmen}

\textbf{Einwilligung zu den Teilnahmebedingungen}

Mit unserer Unterschrift bestätigen wird, dass wir die beigelegten
Teilnahmebedingungen (Stand 06.12.2024) gelesen und verstanden haben. Diesen Bedingungen
stimmen wir zu. Wir versichern, jegliche Änderungen an oben angegebenen
Daten umgehend den Jugendleiter*innen mitzuteilen.


\yesno{Wir stimmen der Bildnutzung in den Sektionsmedien gemäß
Teilnahmebedingungen zu}{bildnutzung}
\yesno{Unser Kind darf sich selbstständig von der Gruppenstunde abmelden.}{abmelden}

\vspace{10pt}
\vfill

\field{Ort, Datum}{0.28\linewidth}{}\hfill
\fieldd{Unterschrift Kind}{0.28\linewidth}{}\hfill
\fieldd{Unterschrift des/der Erziehungsberechtigten}{0.4\linewidth}{}

Bitte diese Seite ausgefüllt und unterschrieben im Registrierungsformular hochladen.
\end{Form}

\newpage

\textbf{\large  Informationen und Bedingungen zur Teilnahme an den
JDAV-Jugendgruppen der DAV-Sektion Heidelberg}

{\small Stand: 06.12.2024}


Liebes neues Jugendgruppenmitglied,

herzlich willkommen als neues aktives Mitglied unserer Sektionsjugend!
Das Zentrum unserer Jugendarbeit sind die wöchentlichen Gruppenstunden,
in denen wir in einer festen Gruppe gemeinsam Klettern oder Fahrrad fahren, Spiele spielen
und Spaß haben. Uns ist wichtig, dass du regelmäßig dabei bist und
deinen Jugendleiter*innen immer bescheid gibst, wenn du mal nicht kommen
kannst.

Zu unserer Jugendarbeit gehören aber auch regelmäßig Ausfahrten und
Veranstaltungen rund um die Kletterhalle. Ausfahrten können ein- oder
mehrtägige Ausflüge zum Klettern in der Region, aber auch in die
Mittelgebirge oder Alpen sein, wo wir gemeinsam Wandern, Bergsteigen,
Klettern... Veranstaltungen in der Kletterhalle können zum Beispiel
gemeinsame Übernachtungen in der Gruppe oder größere Aktivitäten mit der
ganzen Sektionsjugend sein, z.B. unsere jährliche Jugendvollversammlung.
Hier treffen wir basisdemokratisch Entscheidungen über die Zukunft der
Jugendarbeit und wählen Stellvertreter*innen, die die Jugendarbeit
organisieren. Auch du hast hier eine Stimme! Wir freuen uns, wenn du bei
möglichst vielen Aktivitäten dabei sein kannst.

Deine Jugendleiter*innen
\\[3mm]
\_\_
\\[3mm]
Liebe Eltern,

Bitte lesen Sie diese Teilnahmebedingungen für unsere Jugendarbeit genau
und bestätigen Ihr Einverständnis mit Ihrer Unterschrift auf unserer
Einverständniserklärung.

\textbf{Durchgeführte Aktivitäten}

In unserer Jugendgruppe bieten wir im weitesten Sinne verschiedene
Aktivitäten und Aktionen an, zu denen wir die Jugendlichen vorher
entsprechend einweisen, dazu gehören u.a.:

\begin{itemize}
    \setlength\itemsep{1pt}
\item
  Jugendgruppenstunden mit Spielen, Übungen usw.
\item
  Outdoorspiele in verschiedenen Formen
\item
  Klettern an künstlichen Anlagen und natürlichen Felsen
\item
  Wanderungen, Bachwanderungen oder Klettersteigbegehungen
\item Fahrradtouren z.B. mit dem Mountainbike
\item
  natursportliche Aktivität mit den dort typischen Bedingungen und
  Risiken
\item
  Gruppenübergreifende Ausfahrten und Veranstaltungen der JDAV
  Heidelberg.
\end{itemize}

Während aller Aktivitäten werden die Jugendlichen von unseren
Jugendleiter*innen, die eine entsprechende Qualifizierung (DAV
Jugendleiterausbildung / DAV Fachübungsleiter usw.) vorweisen können,
betreut und beaufsichtigt.

Um teilnehmen zu können, müssen die Jugendlichen Mitglied in der Sektion
Heidelberg des DAV sein und eine von den Erziehungsberechtigten
unterschriebene Einverständniserklärung abgeben. Wir erwarten eine
möglichst regelmäßige Teilnahme an den Gruppenstunden. Bei wiederholt
unentschuldigtem Fehlen (ab 3 mal) kann es zum Ausschluss aus der Gruppe
kommen.

Mit Ihrer Unterschrift wird außerdem die aktuelle Fassung der
\textbf{Benutzer- und Hallenordnung der Kletterhalle} der Sektion
Heidelberg (Harbigweg 20, 69124 Heidelberg), anerkannt. Die Benutzer-
und Hallenordnung hängt ständig in der Kletterhalle aus. Den Anweisungen
des Personals ist Folge zu leisten. Die Gruppenstunden finden
gelegentlich auch \textbf{außerhalb des Vereinszentrums} im Harbigweg
20, 69124 Heidelberg, statt. Mit Ihrer Unterschrift wird auch
zugestimmt, dass diese Gruppenstunden zu anderen Zeiten und an anderen
Orten durchgeführt werden können.

\textbf{Haftung}

Mit Ihrer Unterschrift erklären Sie sich damit einverstanden, dass Ihr
Kind an den oben genannten
Aktivitäten im weitesten Sinne teilnehmen darf. Sie sind sich bewusst,
dass natursportliche und vor allem klettersportliche Aktivitäten mit
Risiken verbunden sind, die sich nicht vollständig ausschließen lassen.
Sie erkennen damit an, dass die Sektion Heidelberg und ihre
verantwortlichen ehrenamtlichen
Jugendleiter*innen - soweit gesetzlich zulässig - von jeglicher Haftung
sowohl im Grunde als auch
der Höhe nach freigestellt werden, die über den im Rahmen der
Mitgliedschaft im DAV,
sowie für die ehrenamtliche Tätigkeit bestehenden Versicherungsschutz
hinausgehen. Dies
gilt nicht für die Verursachung von Unfällen durch Vorsatz oder grobe
Fahrlässigkeit.

Weitere Informationen zum DAV und dessen Jugendarbeit aber auch z.B. zu
den
Versicherungsbedingungen oder aktuellen Informationen finden Sie unter
\href{http://www.alpenverein.de}{www.alpenverein.de}

\textbf{Weitere Informationen zu Ihrem Kind}

Sie verpflichten sich, die Jugendleiter*innen insbesondere über folgende
persönliche Einschränkungen und zu informieren (über die angehängte
Einverständniserklärung):

\begin{itemize}
    \setlength\itemsep{1pt}
\item
  relevante allergische Reaktionen
\item
  relevante asthmatische Erkrankungen oder Beschwerden
\item
  relevante Herzkreislauferkrankungen oder Beschwerden
\item
  Diabetes
\item
  Epilepsie oder relevante Nervenerkrankungen
\item
  relevante orthopädische Probleme (Bandscheibenvorfall, Verrenkungen,
  Knochenbrüche\ldots)
\item
  relevante Infektionskrankheiten
\item
  ADHS oder sonstige Aufmerksamkeitsstörungen
\item
  Einnahme von dringend notwendigen Medikamenten
\item
  relevante Suchterkrankungen, psychische Erkrankungen oder
  Einschränkungen
\item
  sonstige relevante Beeinträchtigungen oder Erkrankungen (Bspw.
  Inkontinenz)
\end{itemize}

Die JDAV weist darauf hin, dass durchgeführte Aktivitäten auch im Freien
in Risikogebieten für
von Zecken übertragene FSME oder Borreliose stattfinden können
(Rhein-Neckar Kreis gehört zu den Risikogebieten). Eine Impfung wird
empfohlen.

Ergeben sich Veränderungen bei den angegebenen Informationen, dann
müssen die Jugendleiter*innen umgehend darüber informiert werden.

\textbf{Datenschutz}

Sie sind damit einverstanden, dass die JDAV Heidelberg die von Ihnen
angegebenen Daten elektronisch erfasst, verarbeitet und speichert. Diese
Datenverarbeitung erfolgt in Rechenzentren in der Schweiz oder Staaten
des Europäischen Wirtschaftsraums (EWR). Nach einer Beendigung der
Mitgliedschaft werden Ihre personenbezogenen Daten gelöscht, soweit sie
nicht, entsprechend den steuerrechtlichen Vorgaben, aufbewahrt werden
müssen. Die Datenverarbeitung erfolgt auf Grundlage des Art. 6 Abs. 1 S.
1 lit. b DSGVO. Die jeweils aktuelle Datenschutzerklärung kann jederzeit
auf der Website unter \href{https://www.jdav-hd.de/datenschutz}{www.jdav-hd.de/datenschutz} abgerufen
werden.

Die Jugendleiter*innen sind im Rahmen ihrer Tätigkeiten über
Sektions-E-Mailadressen nach dem Schema
vorname.nachname@alpenverein-heidelberg.de erreichbar, die Jugendleiter*innen einer Gruppe über
Adressen nach dem Schema leitung.gruppenname@jdav-hd.de. Bitte nutzen Sie
für die Kontaktaufnahme vorrangig diese E-Mailadressen.

Zur Weitergabe der Daten im Rahmen einer Mitgliedschaft im Alpenverein
Heidelberg beachten Sie bitte auch die Datenschutzerklärung der Sektion
unter \href{https://www.alpenverein-heidelberg.de/datenschutz}{www.alpenverein-heidelberg.de/datenschutz}.

\textbf{Abbildungen in Sektionsmedien:}

Selbstverständlich gehören auf eine lebendig gestaltete Homepage sowie
andere Medien der Sektion (Aushänge im Vereinsheim, Sektionsnachrichten,
etc.) auch Berichte und Fotos von Veranstaltungen der Jugend, bei denen
vielleicht auch Ihr Kind dabei ist. Da gemäß § 22 KunstUrhG, Bilder
jedoch nur mit Einwilligung des Abgebildeten bzw. seiner gesetzlichen
Vertreter verbreitet oder öffentlich zur Schau gestellt werden dürfen,
bitten wir Sie um Erlaubnis. Mit Ihrer Einverständniserklärung
erleichtern Sie uns wesentlich die Arbeit und wir können so unseren
lebendigen Verein gut nach außen darstellen.

Wenn sie in der Einverständniserklärung ihre Einwilligung geben,
erklären Sie sich damit einverstanden, dass von den Jugendleiter*innen
ausgewählte Fotos und Aufnahmen, auf denen Ihr Kind zu sehen ist, mit
den anderen Kindern geteilt werden, auf der Homepage, sowie im
Sektionsheft der DAV-Sektion Heidelberg und allen anderen Vereinsmedien
erscheinen und veröffentlicht werden dürfen. Diese Einwilligung kann
jederzeit widerrufen werden.


\end{document}
